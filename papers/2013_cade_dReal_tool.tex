\documentclass[envcountsect]{llncs}
%\usepackage{stmaryrd,amsmath,amssymb,newlfont,graphicx,caption,verbatim}
\usepackage{amsmath,amssymb,newlfont,graphicx,caption,verbatim}
\usepackage[ruled,lined,boxed,commentsnumbered,linesnumbered]{algorithm2e}
\newcommand{\dom}{\mathrm{dom}}
\newtheorem{notation}[theorem]{Notation}
\newcommand{\len}{\mathit{len}}
\newcommand{\poly}{\mathsf{poly}}

%\setlength{\textwidth}{5.7in}
%\setlength{\textheight}{8.2in}
%\setlength{\topmargin}{0in}
%\setlength{\oddsidemargin}{.4in}
%\setlength{\evensidemargin}{.4in}

\title{The {\tt dReal} Tool}
\author{Sicun Gao \and Soonho Kong \and Edmund M. Clarke}
\institute{Carnegie Mellon University, Pittsburgh, PA 15213}

\begin{document}
\maketitle

\begin{abstract}
We present the tool {\tt dReal}, an SMT solver for bounded
nonlinear formulas over the reals in the framework of $\delta$-complete decision
procedures. The tool handles various nonlinear functions including polynomials,
trignometric functions, exponential functions, as well as  many ordinary
differential equations. 
{\tt dReal} is numerically-driven, but provides the following correctness
guarantees: if a formula is ``unsat'', then it is indeed unsatisfiable and a proof
of unsatisfiability is constructed; if a
formula is ``$\delta$-sat'', then under some $\delta$-bounded syntactic perturbation it would
be satisfiable, with a witness solution returned. We show its promising
performance on various nonlinear benchmarks. 
\end{abstract}

\section{Introduction}

Given a first-order signature $\mathcal{L}$ and a 
structure $\mathcal{M}$, the {\em Satisfiability Modulo Theories} (SMT) problem
asks whether a quantifier-free $\mathcal{L}$-formula is satisfiable over
$\mathcal{M}$. SMT problems in theories over the real
numbers are of particular interest because of their applications in formal
verification, theorem proving, control theory, etc. While efficient
algorithms~\cite{linear06} exist for deciding SMT problems with only linear real
arithmetic, practical problems normally contain nonlinear polynomials,
transcendental functions, and differential equations. Solving formulas with
these functions is inherently intractable.

Our recent work on $\delta$-complete decision procedures
provided a new framework for handling general nonlinear formulas~\cite{}. Fix
any posiive rational number $\delta$ and any input formula $\varphi$, a
$\delta$-complete decision procedures returns on of the following answers:
\begin{itemize}
 \item {\sf unsat}: $\varphi$ is unsatisfiable.
 \item {\sf $\delta$-sat}: $\varphi^{\delta}$ is satisfiable.
\end{itemize}
Here, $\varphi^{\delta}$ is a syntactic variation of $\varphi$ that encodes a
notion of numerical perturbation on the logic formula. (We will review more
details in Section \ref{review}.) In other words, we allow such a procedure to
give answers with one-sided, $\delta$-bounded errors. Because of the
relaxations, $\delta$-complete decision procedures can fully exploit the
power of various numerical algorithms and can be highly scalable on
nonlinear problems. The $\delta$-completeness guarantees can be suitably
applied to solving various correctness-critical problems in formal verification
and theorem proving~\cite{}.

\section{$\delta$-Complete Decision Procedures}




\section{Tool Design}

\section{Input Format and Usage}

\section{Examples and Results}


\bibliographystyle{abbrv}
\bibliography{tau}
\end{document}

